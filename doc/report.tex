\documentclass[11pt,a4paper]{article}
\usepackage[utf8]{inputenc}
\usepackage[T1]{fontenc}
\usepackage{hyperref}
\usepackage{titlesec}
\usepackage{geometry}
\usepackage{listings}
\usepackage{xcolor}
\usepackage{graphicx}
\usepackage{amsmath}

\geometry{margin=1in}

\definecolor{codegreen}{rgb}{0,0.6,0}
\definecolor{codegray}{rgb}{0.5,0.5,0.5}
\definecolor{codepurple}{rgb}{0.58,0,0.82}
\definecolor{backcolour}{rgb}{0.95,0.95,0.92}

\lstset{
    backgroundcolor=\color{backcolour},   
    commentstyle=\color{codegreen},
    keywordstyle=\color{magenta},
    numberstyle=\tiny\color{codegray},
    stringstyle=\color{codepurple},
    basicstyle=\ttfamily\footnotesize,
    breakatwhitespace=false,         
    breaklines=true,                 
    captionpos=b,                    
    keepspaces=true,                 
    numbers=left,                    
    numbersep=5pt,                  
    showspaces=false,                
    showstringspaces=false,
    showtabs=false,                  
    tabsize=2
}

\title{\textbf{Redis-from-Scratch: High-Performance C++ Implementation}}
\author{Antigravity Technical Report}
\date{\today}

\begin{document}

\maketitle

\section{Introduction}
This report details the implementation of a high-performance Redis-like server written in C++. The project focuses on modern C++ paradigms, memory efficiency through zero-copy architectures, and reliable persistence mechanisms.

\section{System Architecture}
The server employs a decoupled \textbf{Producer-Consumer} architecture, visualized in Figure \ref{fig:architecture}.

\begin{figure}[h]
    \centering
    % Ensure you have generated architecture_diagram.pdf from the mermaid file
    % \includegraphics[width=0.9\textwidth]{architecture_diagram.pdf} 
    \includegraphics[width=0.9\textwidth]{architecture_diagram}
    \caption{System Architecture Class Diagram}
    \label{fig:architecture}
\end{figure}

The core components are:
\begin{itemize}
    \item \textbf{Server}: The central coordinator that initializes the networking stack and the database. It manages a collection of \texttt{Client} objects.
    \item \textbf{TcpServer \& TcpConnection}: Handle the raw socket I/O and event polling (using \texttt{poll}).
    \item \textbf{Client}: Encapsulates the state of a connected client, including its protocol parser (\texttt{RespParser}) and a command queue.
    \item \textbf{Lock-Free Ring Buffer}: A thread-safe queue used by the \texttt{Client} to buffer parsed commands for the main server thread to consume.
    \item \textbf{Database}: The core data store supporting various Redis data types.
\end{itemize}

\section{Core Implementation \& Optimizations}
\subsection{Zero-Copy Data Path}
Memory efficiency is achieved by minimizing data copies throughout the stack:
\begin{itemize}
    \item \textbf{SegmentedBuffer}: Custom buffer management that allows direct socket reads into pre-allocated segments.
    \item \textbf{std::string\_view}: Used for all key lookups and command parsing, pointing directly into the network segments instead of allocating new heap memory.
    \item \textbf{RespValue Variant}: A memory-optimized \texttt{std::variant} that represents all Redis data types with minimal overhead.
\end{itemize}

\subsection{Data Structures}
The database utilizes custom intrusive data structures:
\begin{itemize}
    \item \textbf{Intrusive HashMap}: Supports incremental rehashing to prevent latency spikes during resizing.
    \item \textbf{AVL Tree}: Powers Sorted Sets (\texttt{ZSET}), maintaining balanced order for efficient range queries.
\end{itemize}

\section{Persistence \& The Visitor Pattern}
To ensure data durability, the system implements an \textbf{Append Only File (AOF)} mechanism.

\subsection{Background AOF Rewrite}
AOF files can grow indefinitely. Compaction is handled via the \texttt{BGREWRITEAOF} command:
\begin{itemize}
    \item \textbf{Forking}: The server forks a child process to perform a point-in-time snapshot of the database state.
    \item \textbf{Visitor Pattern}: The database implements an \texttt{accept(IDataVisitor\&)} method. This pattern decouples the internal data structure layouts from the persistence serialization logic. The \texttt{AofRewriteVisitor} traverses the database and writes a compacted set of RESP commands to a temporary file.
\end{itemize}

\section{Roadmap \& Future Work}
\subsection{Pub/Sub Mechanism}
The next major architectural phase will be the implementation of \textbf{Pub/Sub}. This will involve:
\begin{itemize}
    \item A global subscription registry to track client interests in channels.
    \item Efficient message broadcasting from producers to subscribers.
    \item Integration with the multi-threaded I/O layer to handle asynchronous notifications.
\end{itemize}

\subsection{Advanced Features}
Future developments include RDB binary snapshots, \texttt{WATCH/MULTI} transactions, and cluster support.

\section{Conclusion}
The Redis-from-Scratch project demonstrates that zero-copy lookups and efficient thread-safety patterns can significantly enhance the throughput of networked data stores in C++.

\end{document}
